% Options for packages loaded elsewhere
\PassOptionsToPackage{unicode}{hyperref}
\PassOptionsToPackage{hyphens}{url}
%
\documentclass[
]{article}
\usepackage{amsmath,amssymb}
\usepackage{lmodern}
\usepackage{iftex}
\ifPDFTeX
  \usepackage[T1, T2A]{fontenc}
  \usepackage[utf8]{inputenc}
  \usepackage{textcomp} % provide euro and other symbols
\else % if luatex or xetex
  \usepackage{unicode-math}
  \defaultfontfeatures{Scale=MatchLowercase}
  \defaultfontfeatures[\rmfamily]{Ligatures=TeX,Scale=1}
\fi
% Use upquote if available, for straight quotes in verbatim environments
\IfFileExists{upquote.sty}{\usepackage{upquote}}{}
\IfFileExists{microtype.sty}{% use microtype if available
  \usepackage[]{microtype}
  \UseMicrotypeSet[protrusion]{basicmath} % disable protrusion for tt fonts
}{}
\makeatletter
\@ifundefined{KOMAClassName}{% if non-KOMA class
  \IfFileExists{parskip.sty}{%
    \usepackage{parskip}
  }{% else
    \setlength{\parindent}{0pt}
    \setlength{\parskip}{6pt plus 2pt minus 1pt}}
}{% if KOMA class
  \KOMAoptions{parskip=half}}
\makeatother
\usepackage{xcolor}
\usepackage[margin=1in]{geometry}
\usepackage{graphicx}
\makeatletter
\def\maxwidth{\ifdim\Gin@nat@width>\linewidth\linewidth\else\Gin@nat@width\fi}
\def\maxheight{\ifdim\Gin@nat@height>\textheight\textheight\else\Gin@nat@height\fi}
\makeatother
% Scale images if necessary, so that they will not overflow the page
% margins by default, and it is still possible to overwrite the defaults
% using explicit options in \includegraphics[width, height, ...]{}
\setkeys{Gin}{width=\maxwidth,height=\maxheight,keepaspectratio}
% Set default figure placement to htbp
\makeatletter
\def\fps@figure{htbp}
\makeatother
\setlength{\emergencystretch}{3em} % prevent overfull lines
\providecommand{\tightlist}{%
  \setlength{\itemsep}{0pt}\setlength{\parskip}{0pt}}
\setcounter{secnumdepth}{-\maxdimen} % remove section numbering
\ifLuaTeX
\usepackage[bidi=basic]{babel}
\else
\usepackage[bidi=default]{babel}
\fi
\babelprovide[main,import]{russian}
% get rid of language-specific shorthands (see #6817):
\let\LanguageShortHands\languageshorthands
\def\languageshorthands#1{}
\ifLuaTeX
  \usepackage{selnolig}  % disable illegal ligatures
\fi
\IfFileExists{bookmark.sty}{\usepackage{bookmark}}{\usepackage{hyperref}}
\IfFileExists{xurl.sty}{\usepackage{xurl}}{} % add URL line breaks if available
\urlstyle{same} % disable monospaced font for URLs
\hypersetup{
  pdflang={russian},
  hidelinks,
  pdfcreator={LaTeX via pandoc}}

\author{}
\date{\vspace{-2.5em}}

\begin{document}

\newcommand{\redline}{\hspace*{1cm}}
\thispagestyle{empty}
\begin{center}
Санкт-Петербургский Государственный Университет\\
Прикладная математика и информатика\\
\vspace*{50mm} Отчёт по учебной практике 2 (научно-исследовательской работе)(семестр 4)\\
«Метод SSA для временных рядов»
\end{center}
\vspace*{40mm}
\begin{flushright}
Выполнил:\\
Яковлев Денис Михайлович\\
группа 21.Б06-мм\\ \mbox{} \\
Научный руководитель:\\
К.ф-м.н., доцент\\
Голяндина Нина Эдуардовна\\
Кафедра статистического моделирования\\
\end{flushright}
\vspace*{30mm}
\begin{center}
\vspace*{30mm}
Санкт-Петербург\\
2022
\end{center}
\newpage
\thispagestyle{empty}
\tableofcontents
\newpage
\pagenumbering{arabic}

\section{Введение}

Метод SSA (Singular Spectrum Analysis - анализ сингулярного спектра),
или «Гусеница»-SSA - один из методов анализа временных рядов,
охаратеризованный широким спектром преимуществ \textbf{[1, с. 4]}. Так,
например, при отсутствии предположений о:

\begin{itemize}
    \item Стационарности ряда;
    \item Знаний модели тренда;
    \item Сведений о наличии в ряде периодических составляющих и их периодах;
\end{itemize}

Метод SSA может решать различные задачи: выделение тренда, периодик,
сглаживание ряда, построение полного разложения ряда в сумму тренда,
периодик и шума, и так далее. \textbackslash{} Объясним суть базового
алгоритма метода «Гусеница»-SSA \textbf{[1, с.5]}, выполняющегося в два
этапа - разложение и восстановление:

\subsection{Первый этап: разложение}

Рассмотрим вещественнозначный временной ряд
\(\textit{F} = (f_0, ..., f_{N-1})\) длины \(N > 2\). \textbackslash{}

\subsubsection{Шаг 1: Вложение}

Процедура переводит временной ряд в траекторную матрицу, построенную из
последовательности векторов фиксированного размера. Пусть \(L\) -
некоторое целое число \textit{(длина окна)}, \(1 < L < N\),
\(K = N - L + 1 -\) число окон (векторов \(L\)-вложения) размерности
\(L\) \[X_i = (f_{i-1},..., f_{i+L-2})^T, 1 \leq i \leq K.\]
\(\textit{L}\)\textit{-траекторной матрицей} ряда \(\textit{F}\) назовём
\begin{align}
\textbf{X} = [X_1:...:X_K] = 
\begin{pmatrix}
    f_0 & f_1 & f_2 & ... & f_{K-1} \\
    f_1 & f_2 & f_3 & ... & f_{K} \\
    f_2 & f_3 & f_4 & ... & f_{K+1} \\
    ... & ... & ... & \ddots & ... \\
    f_{L-1} & f_{L} & f_{L+1} & ... & f_{N-1} 
\end{pmatrix}
\end{align} Откуда можно заметить, что траекторная матрица размерности
\(L \times K\) является \textit{ганкелевой матрицей} - имеет одинаковые
элементы на диагоналях \(i + j = const\)

\newpage

\subsubsection{Шаг 2: Сингулярное разложение}

На втором шаге этапа разложения используем
\textit{сингулярное разложение} (SVD - Singular Value Decomposition).
Пусть
\(\textbf{S} = \textbf{XX}^T, \lambda_1, ..., \lambda_L - \textit{собственные числа матрицы } \textbf{S}\);
по построению, все собственные числа - неотрицательные, поэтому
расположим их в порядке неубывания
(\(\lambda_1 \geq ... \geq \lambda_L \geq 0\)) и \$U\_1, \ldots, U\_L -
\$ ортонормированная система \textit{собственных векторов}
(\textit{левых сингулярных векторов}) матрицы \(\textbf{S}\).
\textbackslash{} Теперь, пусть \(d = max{i: \lambda_i > 0}\). Тогда,
если обозначить
\[V_i = \textbf{X}^TU_i/\sqrt{\lambda_i} - \textit{правый сингулярный (факторный) вектор}, i = 1, 2, ..., d\]
то сингулярное разложение матрицы \(\textbf{X}\) можно записать как
\begin{align}
    \textbf{X} = \textbf{X}_1 + ... + \textbf{X}_d
\end{align} где \(\textbf{X}_i = \lambda_iU_iV_i^T\) -
\textit{элементарная матрица}, так как имеет ранг 1. \textbackslash{}
Набор (\(\sqrt{\lambda_i}, U_i, V_i\)) -
\(i\)\textit{-я собственная тройка} сингулярного разложения (2).

\subsection{Второй этап: восстановление}
\subsubsection{Шаг 1: Группировка}

Процедура группировки делит множество индексов \(\{1, ..., d\}\) на
\(m\) непересекающихся подмножеств \(I_1, ..., I_m\). \textbackslash{}
Если положить, что
\(I = {i_1, ..., i_p}, \forall k = 1, ...,p, i_k \in {1, ..., d}\), то
\textit{результирующая матрица} \(\textbf{X}_I\) имеет вид
\[\textbf{X}_I = \textbf{X}_{i_1} + ... + \textbf{X}_{i_p}\] Таким
образом, разложение (2) можно представить в сгруппированном виде:
\begin{align}
    \textbf{X} = \textbf{X}_{I_1} + ... + \textbf{X}_{I_m}
\end{align} \textit{Группировка собственных троек} --- процедура выбора
множеств \(I_1, ..., I_m\).

\subsubsection{Шаг 2: Диагональное усреднение}

На последнем шаге алгоритма каждая сгруппированная матрица из разложения
(3) переводится в новый ряд длины \(N\). \textbackslash{} Для этого
рассмотрим полученную результирующую матрицу, например, \(\textbf{X}_I\)
размера \(L \times K\) с элементами
\(x_{ij}, 1 \leq i \leq L, 1 \leq j \leq K\). Для общности работы
алгоритма транспонируем результирующую матрицу, если \(L > K\). В общем
виде положим \(L^* = min(L, K), K^* = max(L, K)\) и
\(x_{ij}^* = x_{ij} \ (L \leq K), x_{ij}^* = x_{ji} \ (L > K)\), где
\(\textbf{X}^*_I =(x_{ij}^*)^{L^*,K^*}_{i,j=1}\).
\newpage \textit{Диагональное усреднение} переводит матрицу
\(\textbf{X}^*_I\) в ряд
\(\tilde{F} = (\tilde{f}_0, ..., \tilde{f}_{N-1})\) по следующей
формуле: \begin{align}
    \Tilde{f}_k = 
    \begin{cases}
        \frac{1}{k+1}\sum\limits^{k+1}_{m=1}x^*_{m, k - m + 2}, & \textit{для } 0 \leq k < L^*-1, \\
        \frac{1}{L^*}\sum\limits^{L^*}_{m=1}x^*_{m, k - m + 2}, & \textit{для } L^* - 1 \leq k < K^*, \\
        \frac{1}{N-k}\sum\limits^{N-K^*+1}_{m=k-K^*+2}x^*_{m, k - m + 2}, & \textit{для } K^* \leq k < N,
    \end{cases}
\end{align} Применяя диагональное усреднение (4) к результирующим
матрицам \(\textbf{X}_{I_k}\), получаем ряды
\(\tilde{F}^{(k)} = (\tilde{f}^{(k)}_0, ..., \tilde{f}^{(k)}_{N-1}\).
Следовательно, исходный ряд \((f_0, ..., f_{N-1})\) раскладывается в
сумму \(m\) рядов: \begin{align}
    f_n = \sum\limits^m_{k=1}\tilde{f}^{(k)}_n, n = 0, 2, ..., N-1.
\end{align} Таким образом, в ходе исполнения базового алгоритма, на
основе построенной траекторной матрицы после сингулярного разложения
группируются наборы собственных троек, характеризующие тренд, периодику
или шум. Поэтому, целью этого отчёта являются ознакомление с критериями
группировки собственных троек, а также описание типов разделимости для
произвольной траекторной матрицы на примерах.

\newpage
\section{Теоретические задачи}
\subsection{Задача 1. Типы разделимости и их отсутствие.}

\textit{Продемонстрируйте понятия сильной и/или слабой разделимости, а также их отсутствия, на разных рядах, с шумом и без.}

\subsubsection{Пример 1. Сильная и слабая разделимость}

Рассмотрим следующие последовательности: \begin{align}
    f^{(1)}_n = \alpha cos(2\pi \omega n ), \ \alpha \not= 0, \ 0 \leq \omega \leq 1/2\tag{*} \\
    f^{(2)}_n = \beta, \beta \ \not= 0 \tag{**}
\end{align} Пусть длина ряда \(N = 7\), длина окна \(L = 4\), число окон
(векторов вложения) \(K = N - L + 1 = 4\), ряды
\(F_N^{(1)} = (f_0^{(1)}, ..., f_{N-1}^{(1)}), F_N^{(2)} = (f_0^{(2)}, ..., f_{N-1}^{(2)})\).
Воспользуемся следующим предложением \textbf{[2, с.14]}:
\textbackslash{}\redline \textbf{Предложение 1.}
\textit{ Пусть }\(K = N-L+1.\) \textit{Ряды }\(F_N^{(1)}\) \textit{ и }
\(F_N^{(2)}\) \textit{ слабо разделимы тогда и только тогда, когда}
\[f_k^{(1)}f_m^{(2)} +\dots+f^{(1)}_{k+L-1}f^{(1)}_{m+L-1} = 0, \ 0 \leq k, m \leq K-1\]
\[f_k^{(1)}f_m^{(2)} +\dots+f^{(1)}_{k+K-1}f^{(1)}_{m+K-1} = 0, \ 0 \leq k, m \leq L-1\]
Для матриц \[F^{(1)}_N \longleftrightarrow \textbf{X}^{(1)} = 
\begin{pmatrix}
    \alpha & \alpha cos(2\pi \omega n) & \alpha cos(4\pi \omega n) & \alpha cos(6\pi \omega n) \\
    \alpha cos(2\pi \omega n) & \alpha cos(4\pi \omega n) & \alpha cos(6\pi \omega n) & \alpha cos(8\pi \omega n) \\
    \alpha cos(4\pi \omega n) & \alpha cos(6\pi \omega n) & \alpha cos(8\pi \omega n) & \alpha cos(10\pi \omega n) \\
    \alpha cos(6\pi \omega n) & \alpha cos(8\pi \omega n) & \alpha cos(10\pi \omega n) & \alpha cos(12\pi \omega n) \\
\end{pmatrix}
\] \[
    F^{(2)}_N \longleftrightarrow \textbf{X}^{(2)} = 
\begin{pmatrix}
    \beta & \beta & \beta & \beta \\
    \beta & \beta & \beta & \beta \\
    \beta & \beta & \beta & \beta \\
    \beta & \beta & \beta & \beta \\
\end{pmatrix}
\] По предложению 1 ряды \(F_N^{(1)}, F_N^{(2)}\) слабо разделимы, если
они удовлетворяют следующей системе уравнений:

\begin{center}
\begin{align}
\begin{cases}
    \alpha\beta(1 + cos(2\pi \omega n) + cos(4 \pi \omega n) + cos(6\pi \omega n)) = 0 \\
    \alpha\beta(cos(2 \pi \omega n) + cos(4\pi \omega n) + cos(6 \pi \omega n) + cos(8\pi \omega n)) = 0 \\
    \alpha\beta(cos(4 \pi \omega n) + cos(6\pi \omega n) + cos(8 \pi \omega n) + cos(10\pi \omega n)) = 0 \\
    \alpha\beta(cos(6 \pi \omega n) + cos(8\pi \omega n) + cos(10 \pi \omega n) + cos(12\pi \omega n)) = 0 \\
\end{cases}
\end{align}
\end{center}

Это условие выполняется, если, например, положить \(\omega = 1/4\).
\newpage Обратим внимание на выбор коэффициента \(\alpha\) для элемента
периодического ряда. Всего у \(\textbf{X}^{(1)}\) одно сингулярное число
\(\sqrt{\lambda^{(1)}} = 2\alpha\) кратности 2, в качестве левых и
правых сингулярных векторов:
\(U_1 = V_1 = (1, 0, 1, 0)^T/\sqrt{2}, U_2 = V_2 = (0, 1, 0, 1)^T/\sqrt{2}.\)
У \(\textbf{X}^{(2)}\) одно сингулярное число
\(\sqrt{\lambda^{(2)}} = 0\) кратности 1, в качестве левых и правых
сингулярных векторов: \(U_3 = V_3 = (1, 1, 1, 1)^T/2.\) Тогда

\begin{align*}
    \textbf{X} & = 
    \begin{pmatrix}
        \beta + \alpha & \beta & \beta - \alpha & \beta \\
        \beta & \beta - \alpha & \beta & \beta + \alpha \\
        \beta - \alpha & \beta & \beta + \alpha & \beta \\
        \beta & \beta + \alpha & \beta & \beta - \alpha \\
    \end{pmatrix}
     \\ & = 2\alpha U_1V_1^T + 2\alpha U_2V_2^T + 4\beta U_3V_3^T = \textbf{X}^{(1)} + \textbf{X}^{(2)}
\end{align*}

По определению \textbf{[2, с.15]}, сильная разделимость достигается в
том случае, когда
\(\sqrt{\lambda^{(1)}} \not= \sqrt{\lambda^{(2)}} \ (\alpha \not = 2\beta).\)
Это значит, что сингулярное разложение матрицы \(\textbf{X}\) однозначно
определено (с точностью до сингулярных векторов кратных сингулярных
чисел). Если же \(\alpha = 2\beta\), то

\begin{align*}
    \textbf{X} & =
    \begin{pmatrix}
        3\beta & \beta & -\beta & \beta \\
        \beta & -\beta & \beta & 3\beta \\
        -\beta & \beta & 3\beta & \beta \\
        \beta & 3\beta & \beta & -\beta \\
    \end{pmatrix}
    \\ & = 4\beta U_1V_1^T + 4\beta U_2V_2^T + 4\beta U_3V_3^T
\end{align*} \textbackslash{} Где \$U\_1, U\_2, U\_3 \$ ---
ортонормированный набор векторов в
\(\mathbb{R}^4, \ V_i = X^TU_i/4\beta, \ i = 1,2,3.\) Из всех возможных
наборов подходящим является только одно множество наборов:
\(U_3 = V_3 = (1, 1, 1, 1)^T/2, U_1, U_2 -\) произвольная пара
ортонормированных векторов из сингулярного разложения
\(\textbf{X}^{(1)}\). \textbackslash{} В то же время, показанный пример
позволяет сформулировать и проверить на достоверность достаточные и
необходимые условия отделимости произвольного ряда \(F^{(2)}_N\) от
константного ряда \(F^{(1)}_N\) \textbf{[2, с.17]}:

\begin{enumerate}
    \item Ряд $F^{(2)}_N$ имеет целый период $T$; $L/T, K/T - $ целые;
    \item $f_0^{(2)}+... + f_{T-1}^{(2)} = 0$
\end{enumerate}

Поясним, для чего нужно каждое из условий. \textbackslash{}
\redlineПервое условие позволяет построить систему линейных уравнений
вида (6), где число слагаемых у третьего множителя кратно \(T\).
\textbackslash{} \redlineВторое условие приводит к выполнению слабой
разделимости (при первом условии), так как из-за наличия периода в
\(F^{(2)}, \forall i: 0 \leq i \leq N - T: f^{(2)}_{T+i}=f^{(2)}_i\)
сумма в третьем множителе всегда будет равняться нулю.

\subsubsection{Пример 2. Отсутствие разделимости}

\redline Следует обратить внимание и на случаи, когда разделимость не
имеет места. Приведём простой пример неотделимости двух рядов через
константные ряды. Пусть имеется два константных ряда:
\(F^{(1)}_N=(\alpha, \alpha, ..., \alpha), F^{(2)}_N=(\beta, \beta, ..., \beta), \ \alpha, \beta \not = 0\).
По необходимому и достаточному условию слабой разделимости, приняв
\(1 < L < N, K = N - L + 1\), проверим, чему равно скалярное
произведение отрезков длины L, K от рядов \(F^{(1)}_N, F^{(2)}_N\).
Тогда \begin{align*}
    f^{(1)}_kf^{(2)}_m + ... + f^{(1)}_{k+L-1}f^{(2)}_{m+L-1} = \alpha \beta + ... + \alpha \beta = \alpha \beta L \ (1 \leq k, m \leq K-1)
    \\
    f^{(1)}_kf^{(2)}_m +... + f^{(1)}_{k+K-1}f^{(2)}_{m+K-1} = \alpha \beta + ... + \alpha \beta = \alpha \beta K \ (1 \leq k, m \leq L-1)
\end{align*} Отсюда следует, что между константными рядами не существует
слабой разделимости, а, значит, сингулярное разложение произвольной
траекторной матрицы в виде суммы от двух константных рядов.

\subsubsection{Пример 3. Разделимость с шумом}

Покажем визуально на основе первого примера. Возьмём тот случай из
примера 1, когда сингулярные числа периодического \((F^{(1)}_N)\) и
константного \((F^{(2)}_N)\) рядов совпадают. Выпишем явно: \[
F^{(1)}_N \longleftrightarrow X^{(1)}=
\begin{pmatrix}
    2 & 0 & -2& 0 \\
    0 & -2 & 0 & 2 \\
    -2& 0 & 2 & 0 \\
    0 & 2& 0 & -2 \\
\end{pmatrix}
\] \[
F^{(2)}_N \longleftrightarrow X^{(2)} =
\begin{pmatrix}
    1 & 1 & 1 & 1 \\
    1 & 1 & 1 & 1 \\
    1 & 1 & 1 & 1 \\
    1 & 1 & 1 & 1 \\
\end{pmatrix}
\] Объединив их в единую траекторную матрицу \(\textbf{X}\), рассмотрим
её собственные тройки \newpage Можно заметить, что на первом рисунке три
сингулярных числа имеют одинаковые значения, из-за чего невозможно точно
интерпретировать их, а также определить периодический и константный
ряды, входящие в него. \textbackslash{} На втором рисунке найти
закономерности стало проще: первое сингулярное число выделяется среди
других и может восприниматься как тренд этой траекторной матрицы. В то
же время, следующие два сингулярных числа близки друг к другу, что даёт
возможность воспринимать их как периодику.

\section{Практическое применение}
\subsection{Формулировка}

В качестве одной из задач прикладного характера предлагается исследовать
ежемесячное измерение средней температуры в Санкт-Петербурге с 1805 по
2021 при помощи метода SSA. Поступившая на вход таблица данных - матрица
с датами в первом столбце и средними температурными измерениями для
каждого месяца (начиная с января) в следующих двенадцати столбцах
соответственно. Тогда полученную матрицу без столбца дат можно
представить как ряд, в котором последовательно записаны изменения
слева-направо, сверху-вниз. Назовём временной ряд \(F\).
\textbackslash{} Перед тем, как приступать к следующей части, приведём
несколько предложений и практических замечаний, которые помогут при
интерпретации полученных результатов:

\begin{itemize}
    \item При выборе длины окна [4.2.2, с.42], хотя и число ненулевых сингулярных значений, скорее всего будет небольшим в сравнение с самой длиной временного ряда, следует брать L как можно ближе к половине, чтобы достигнуть лучшей разделимости (на основе асимптотической разделимости) \textbf{[2, с.19]};
    \item Чтобы извлечь тренд из ряда, нужно собрать все собственные тройки с медленно меняющимися сингулярными векторами;
    \item В общем случае выделения периодической компоненты $F^{(1)}$ такого, что
    \begin{align}
        f_n^{(1)} = \sum\limits^{|T/2|}_{k=1}\sqrt{a^2_k + b^2_k}cos(2\pi k n /T + \psi_k)
    \end{align}
    из ряда $F$, чтобы определить все его собственные тройки, должны выполняться следующие условия
    \begin{enumerate}[(a)]
        \item Ряд $F^{(1)} -$ (приближённо) строго отделим от $F^{(2)}$ в сумме $F = F^{(1)} + F^{(2)}$ при длине окна $L$ 
        \item Все ненулевый мощности $a^2_k+b^2_k$ в разложении (7) различны
        \item $L \gg T$ 
    \end{enumerate}
\newpage
    \item Для выделения компоненты шума, как и любой другой компоненты, можно воспользоваться сингулярными числами, учитывая поведение сингулярных чисел для определённых типов рядов. Например, периодическая компонента порождается двумя собственные тройки с близкими друг к другу сингулярными числами, только если частота не равняется 0 или 1/2.
\end{itemize}

\subsection{Анализ результатов}

Исходя из значений норм компонентов, можно предположить, что у
рассматриваемого временного ряда в качестве тренда выступают
экспоненциально-модулированный гармонический ряд (ET1,2) и
экспоненциальный ряд (ET3). \textbackslash{} В качестве периодик можно
выделить следующие: ET(4,5) ET(6,7), ET(8,9), ET(10,11), ET(12,13),
ET(14,15), ET(16,17), ET(18,19), ET(20,21), ET(22,23), ET(24,25),
ET(26), ET(27, 28), ET(29), ET(30, 31). То есть, (ET4-31). В этих
случаях ссылаемся на близость сингулярных чисел друг к другу. То же
самое видно и относительно других, более низких по своему вкладу
собственных троек. Например, ET(32,33), ET(34,35), ET(37,38). Однако эти
собственные тройки, исходя из доли вклада, будут относиться к шуму.
\textbackslash{} Таким образом,

\begin{itemize}
    \item Тренд - ET(1,2), ET(3);
    \item Периодика - ET(4-31);
    \item Шумы - ET(32-50);
\end{itemize}
\newpage
\section{Заключение}

В ходе проведённых работ на языке R был продемонстрирован метод SSA для
временных рядов на основе теоретических и практических задач. При
решении теоретических задач были изучены случаи сильной, слабой
разделимости и её отсутствия. При решении практических задач были
изучены особенности разделения траекторной матрицы на составляющие:
тренды, периодики, шумы.

\newpage
\section{Список литературы}
\begin{enumerate}
    \item \textbf{Метод Гусеница-SSA: анализ временных рядов: Учеб. пособие. / Голяндина Н.Э. - СПб., 2004. - 76 с.}
\end{enumerate}

\end{document}
