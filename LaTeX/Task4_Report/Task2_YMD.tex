\documentclass[specialist,
substylefile = spbu_report.rtx,
subf,href,colorlinks=true, 12pt]{disser}

\usepackage[a4paper,
mag=1000, includefoot,
left=3cm, right=1.5cm, top=2cm, bottom=2cm, headsep=1cm, footskip=1cm]{geometry}
\usepackage[T2A]{fontenc}
\usepackage[utf8]{inputenc}
\usepackage[english, russian]{babel}
\ifpdf\usepackage{epstopdf}\fi

\usepackage{bbm}
\usepackage{algorithm}
\usepackage{algpseudocode}
\usepackage{algorithmicx}
\usepackage{amsmath,amssymb,amsthm,amscd,amsfonts}
\usepackage{cmap}
\usepackage{euscript}
\usepackage{mathdots}
\usepackage{graphicx}
\usepackage[russian]{cleveref}

% Нумерация подсекций в оглавлении
\setcounter{secnumdepth}{2}

% Включать подсекции в оглавление
\setcounter{tocdepth}{3}

\newcommand{\R}{\mathbb{R}}
\begin{document}
	
	% Название организации
	\institution{%
		Санкт-Петербургский государственный университет \\
		Прикладная математика и информатика \\
		Вычислительная стохастика и статистические модели
	}
	
	% Тема
	\topic{\normalfont\scshape Отчёт по задачам на тему\\«Моделирование распределений»}
	
	% Автор
	\author{Яковлев Денис Михайлович\\
		Группа 21.Б04-мм\\
		\textsc{095998@student.spbu.ru}}
	
	% Город и год
	\city{Санкт-Петербург}
	\date{\today}
	
	\maketitle
	\newpage
	\thispagestyle{empty}
	\tableofcontents
	\newpage
	\pagenumbering{arabic}
	\section{Задача. Формулировка}
	\textit{\textbf{(МСВ 309)} Промоделировать 3-мя способами распределение с плотностью:}
	\begin{equation*}
		\rho(x)=C
		\begin{cases}
			2-x, & 0 < x \leqslant 1;\\
			1+x, & 1 < x \leqslant 2;\\
			2e^{-x}, & x > 2;\\
			0, & \textit{иначе}.	
		\end{cases}
	\end{equation*}
	\subsection{Метод обратных функций}
	Найдём функцию распределения $F(x)$:
	\begin{equation*}
		F(x) =C
		\begin{cases}
			0, & x \leqslant 0;\\
			2x - \dfrac{x^2}{2}, & 0 < x \leqslant 1;\\
			x + \dfrac{x^2}{2},&1<x\leqslant 2;\\
			4 + 2e^{-2}- 2e^{-x},&x>2.
		\end{cases}
	\end{equation*}
	Определим, чему равна константа $C$:
	\begin{equation*}
		\lim_{x\rightarrow+\infty}F(x) = C(4+2e^{-2})= 1 \Rightarrow C=\dfrac{1}{4+2e^{-2}}. 
	\end{equation*}
%	Теперь определим обратную функцию распределения, предварительно вычислив обратную плотность распределения $p^{-1}(y)$:
%	\begin{equation*}
%		p^{-1}(y)=
%		\begin{cases}
%			0,& y \leqslant C;\\
%			2 - \dfrac{y}{C},& C<y\leqslant2C;\\
%			\dfrac{y}{C}-1,& 2C<y\leqslant3C;\\
%			-\log\left(\dfrac{y}{2C}\right),& 2Ce^{-2} < y.
%		\end{cases}
%	\end{equation*}
	При $y\in(0,1)$ решим уравнение $y=F(x)$ относительно $x$, чтобы получить обратную функцию $x = F^{-1}(y)$.
	\\ Решим уравнение $C(2x-\dfrac{x^2}{2})=y$ при условии, что $0 < x \leqslant 1$:
	\begin{align*}
			C\left(2x-\dfrac{x^2}{2}\right)=y \Leftrightarrow x_{\pm}=\dfrac{4\pm2\sqrt{2}\sqrt{2 - \dfrac{y}{C}}}{2}\\
			\stackrel{x_+\geqslant2}{\Rightarrow}x_-=2-\sqrt{4 - \dfrac{2y}{C}}\Rightarrow y=C\left(2-\dfrac{(4-2x_-)^2}{8}\right).
	\end{align*}
	Подставляя $x_-=0,~x_-=1$, получаем $y = 0,~ y = \dfrac{3}{2}C$ соответственно.
	\\Далее решим $C(x+\dfrac{x^2}{2}) = y$ и получим:
	\begin{equation*}
		x =-1+\sqrt{1+\dfrac{2y}{C}}.
	\end{equation*}
	При $x=1, ~x=2$ получим $y=\dfrac{3}{2}C,~y=4C$ соответственно.\\
	Осталось решить $C(4+2e^{-2}-2e^{-x})=C(C^{-1}-2e^{-x})=y$ относительно $x$. Получим:
	\begin{equation*}
		x = -\log\left(\dfrac{1-y}{2C}\right)
	\end{equation*}
	при $4C<y<1$. Таким образом, обратная функция распределения:
	\begin{equation*}
		F^{-1}(y) =
		\begin{cases}
			2-\sqrt{4 - \dfrac{2y}{C}},&0<y\leqslant\dfrac{3}{2}C;\\
			-1+\sqrt{1+\dfrac{2y}{C}},&\dfrac{3}{2}C<y\leqslant4C;\\
			-\log\left(\dfrac{1-y}{2C}\right),&4C<y<1
		\end{cases}
	\end{equation*}
	\begin{algorithm}[h]
		\caption{Метод обратных функций}
	\begin{algorithmic}[1]
		\State $\mathrm{Get}(\alpha);~C\gets\frac{1}{4+2e^{-2}}$
		\State $d[0]\gets\frac{3}{2}C;~d[1]\gets4C$
		\If{$\alpha < d[0]$}
			\State $\xi\gets2-\sqrt{2}\sqrt{2 - \dfrac{\alpha}{C}}$
		\ElsIf{$\alpha<d[1]$}
			\State $\xi\gets-1+\sqrt{1+\dfrac{2\alpha}{C}}$
		\Else
			\State $\xi\gets-\log\left(\dfrac{1-\alpha}{2C}\right)$
		\EndIf
	\end{algorithmic}
	\end{algorithm}
	\subsection{Метод декомпозиции}
	Разложим плотность распределения
	\begin{align*}
		\rho(x)=C(2-x)\mathbbm{1}_{(0,1]} + C(1+x)\mathbbm{1}_{(1,2]} + 2Ce^{-x}\mathbbm{1}_{(2,+\infty)}
	\end{align*}
	и представим в виде смеси плотностей:
	\begin{align*}
		\rho_1(x)=\dfrac{2}{3}(2-x)\mathbbm{1}_{(0,1]},\\
		\rho_2(x)=\dfrac{2}{5}(1+x)\mathbbm{1}_{(1,2]},\\
		\rho_3(x)=\dfrac{1}{e^{-2}}~e^{-x}\mathbbm{1}_{(2,+\infty)}
	\end{align*}
	Тогда представим плотность через смесь плотностей $\rho(x)=q_1\rho_1(x)+q_2\rho_2(x)+q_3\rho_3(x)$ такую, что:
	\begin{align*}
		q_1=\dfrac{3}{2}C;\\
		q_2=\dfrac{5}{2}C;\\
		q_3=2e^{-2}C.
	\end{align*}
	При этом $q_1 + q_2 + q_3 = \dfrac{4+2e^{-2}}{4+2e^{-2}}=1$. Для $\rho_1(x),\rho_2(x),\rho_3(x)$ найдём соответствующие им обратные функции распределения:
	\begin{align*}
		F_1(x)=\dfrac{2}{3}(2x-\dfrac{x^2}{2}),~F_1^{-1}(y)=2-\sqrt{2}\sqrt{2 - \dfrac{3y}{2}};\\
		F_2(x)=\dfrac{2}{5}(x+\dfrac{x^2}{2}-\dfrac{3}{2}),~F_2^{-1}(y)=-1+\sqrt{4+5y};\\
		F_3(x)=\dfrac{e^{-2}-e^{-x}}{e^{-2}},~F_3^{-1}(y)=2 -\log(1-y),
	\end{align*}
	где $y\in(0,1)$.
	\begin{algorithm}[h]
		\caption{Метод декомпозиции}
	\begin{algorithmic}[1]
		\State $\mathrm{Get}(\alpha_1, \alpha_2);~C\gets\frac{1}{4+2e^{-2}}$
		\State $q[0]\gets\frac{3}{2}C;~q[1]\gets4C$
		\If{$\alpha_1<q[0]$}
		\State $\xi\gets2-\sqrt{4 - 3\alpha_2}$
		\ElsIf{$\alpha_1<q[1]$}
		\State $\xi\gets-1+\sqrt{4+5\alpha_2}$
		\Else
		\State $\xi\gets2 -\log(1-\alpha_2)$
		\EndIf
	\end{algorithmic}
	\end{algorithm}
	\subsection{Метод отбора}
	В качестве мажорирующего распределения возьмём $q(x) = e^{-x}\mathbbm{1}_{(0,+\infty)},~q\sim\mathrm{EXP}(1)$. Его распределение можно промоделировать следующим образом: $\eta = -\log(\alpha_1)$.
	\\Производная Радона-Никодима $r(x)$:
	\begin{align*}
		r(x)=p(x)/q(x)=C
		\begin{cases}
			0,& x\leqslant0;\\
			(2-x)e^{x},& 0<x\leqslant 1;\\
			(1+x)e^{x},& 1<x\leqslant 2;\\
			2,& 2<x.
		\end{cases}
		\\M = 3Ce^2\approx5.19,~C=(4+2e^{-2})^{-1}\approx0.234,
	\end{align*}
	где $M$ --- мажоранта. Так как проверяем неравенство $r(\eta)<M\alpha_2$, то как $r(x)$, так и $M$ можно разделить на общий множитель $C$. Тогда $M = 3e^2 \approx 22.167$.\\
	Далее, неравенство $r(\eta)<M\alpha_2$ выполняется тогда и только тогда, когда либо
	\begin{equation*}
		0<\eta\leqslant1,~(2-\eta)e^\eta < 3e^2\alpha_2
	\end{equation*}
	либо
	\begin{equation*}
		1<\eta\leqslant2,~(1+\eta)e^\eta < 3e^2\alpha_2
	\end{equation*}
	либо
	\begin{equation*}
		2<\eta,~2<3e^2\alpha_2
	\end{equation*}
	Откуда выходят следующие условия:
	\begin{align}
		\alpha_2&>\dfrac{1}{3}(2-\eta)e^{\eta-2},~0<\eta\leqslant1;\label{eq:1}\\
		\alpha_2&>\dfrac{1}{3}(1+\eta)e^{\eta-2},~1<\eta\leqslant2;\label{eq:2}\\
		\alpha_2&>\dfrac{2}{3e^2},~2<\eta;\label{eq:3}
	\end{align}
%	Складывая \eqref{eq:1} и \eqref{eq:2}, получим:
%	\begin{equation*}
%		2\alpha_2>e^{\eta-2}\Rightarrow\alpha_1\alpha_2>\dfrac{1}{2}e^{-2}
%	\end{equation*}
	Тогда $r(\eta) < M\alpha_2$ совпадает с этими условиями:
	\begin{align}
		e^{-1}&<\alpha_1,~\alpha_2\alpha_1>(2+\log(\alpha_1))M^{-1}\tag{1}
		\\e^{-2}&<\alpha_1\leqslant e^{-1},~\alpha_2\alpha_1>(1-\log(\alpha_1))M^{-1}\tag{2}
		\\0&<\alpha_1\leqslant e^{-2},~\alpha_2>2M^{-1}\tag{3}
	\end{align}
	Поскольку условия получились слишком объёмными, в описанном ниже алгоритме рассматривается случай $r(\eta) \geqslant M\alpha_2$, что соответствует следующим условиям:
	\begin{align*}
		e^{-1}&<\alpha_1,~\alpha_2\alpha_1\leqslant(2+\log(\alpha_1))M^{-1}
		\\e^{-2}&<\alpha_1\leqslant e^{-1},~\alpha_2\alpha_1\leqslant(1-\log(\alpha_1))M^{-1}
		\\0&<\alpha_1\leqslant e^{-2},~\alpha_2\leqslant2M^{-1}
	\end{align*}
	\begin{algorithm}[h]
		\caption{Метод отбора}
		\begin{algorithmic}[1]
			\State $M_1\gets\frac{1}{3e^2};~M_2\gets2M_1;~b_1\gets e^{-1};~b_2\gets b_1*b_1$
			\While {TRUE}
			\State $\mathrm{Get}(\alpha_1,\alpha_2)$
			\If{$b_1 < \alpha_1 ~\mathrm{and}~ \alpha_2\alpha_1\leqslant(2+\log(\alpha_1))M_1$}
			\State $\xi\gets-\log(\alpha_1)$
			\EndIf
			\If{$b_2 < \alpha_1 \leqslant b_1~\mathrm{and}~\alpha_1\alpha_2\leqslant(1-\log(\alpha_1))M_1$}
			\State $\xi\gets-\log(\alpha_1)$
			\State break
			\EndIf
			\If{$\alpha_1\leqslant b_2~\mathrm{and}~\alpha_2\leqslant M_2$}
			\State $\xi\gets-\log(\alpha_1)$
			\State break
			\EndIf
			\EndWhile
		\end{algorithmic}
	\end{algorithm}
\end{document}