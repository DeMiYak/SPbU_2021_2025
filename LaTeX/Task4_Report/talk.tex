\documentclass[notheorems, handout]{beamer}

\usetheme[numbers,totalnumbers,compress, nologo]{Statmod}
\usefonttheme[onlymath]{serif}
\setbeamertemplate{navigation symbols}{}

\mode<handout> {
    \usepackage{pgfpages}
    \setbeameroption{show notes}
    \pgfpagesuselayout{2 on 1}[a4paper, border shrink=5mm]
    \setbeamercolor{note page}{bg=white}
    \setbeamercolor{note title}{bg=gray!10}
    \setbeamercolor{note date}{fg=gray!10}
}

\usepackage[utf8x]{inputenc}
%\usepackage[cp1251]{inputenc}
\usepackage[T2A]{fontenc}
\usepackage[russian]{babel}
\usepackage{tikz}
\usepackage{ragged2e}

\newtheorem{theorem}{Теорема}

\title[Короткая тема]{Длинная тема}

\author{Иванов Иван Иванович, гр.18.Б04-мм}

\institute[Санкт-Петербургский Государственный Университет]{%
    \small
    Санкт-Петербургский государственный университет\\
    Прикладная математика и информатика\\
    Вычислительная стохастика и статистические модели\\
    \vspace{1.25cm}
    Отчет по производственной практике}

\date[Зачет]{Санкт-Петербург, 2021}

\subject{Talks}

\begin{document}

\begin{frame}[plain]
    \titlepage

    \note{Научный руководитель  к.ф.-м.н., доцент Петров П.П.,\\
    кафедра статистического моделирования}
\end{frame}


%\section{Короткая тема}
%\subsection{Общие слова}

\setbeameroption{show notes}

\begin{frame}{Введение}
    Тут какое-то введение.
    Что за задача решается, какое метод используется, какая цель работы.

    \note{
        Текст про введение
    }
\end{frame}

\begin{frame}{Обозначения и известные результаты}
    Обозначения, нужные понятия и нужные известные результаты.

    \note{
        Текст про это
    }
\end{frame}

\begin{frame}{Полученные результаты}
    Тут ваши личные результаты

    \note{
        Текст про это
    }
\end{frame}

\begin{frame}{Заключение}
    Тут какое-то заключение.
    Что сделано, резюме по результатам.

    \note{
        Текст про заключение
    }
\end{frame}

\begin{frame}{Список литературы}
\begin{thebibliography}{3}
\bibitem{main}Самая главная работ
\bibitem{another} Еще одна
\end{thebibliography}    

    \note{
        На данном слайде представлен список основных источников, используемых в моей работе.
    }
\end{frame}

\end{document}
