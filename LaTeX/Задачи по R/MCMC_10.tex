\documentclass{article}

\usepackage[a4paper,
mag=1000, includefoot,
left=3cm, right=1.5cm, top=2cm, bottom=2cm, headsep=1cm, footskip=1cm]{geometry}
\usepackage[T2A]{fontenc}
\usepackage[utf8x]{inputenc}
\usepackage[english, main=russian]{babel}
\ifpdf\usepackage{epstopdf}\fi

\usepackage{amsmath,amssymb,amsthm,amscd,amsfonts}
\usepackage{mathdots}
\usepackage{graphicx}
\usepackage{algorithm}
\usepackage{algpseudocode}
\usepackage{algorithmicx}
\usepackage{bbm}
\usepackage{hyperref}

\newcommand\norm[1]{\left\|#1\right\|}
\DeclareMathOperator\R{\mathbb{R}}
\DeclareMathOperator\N{\mathbb{N}}
\DeclareMathOperator\rank{\textrm{rank}\,}


\title{Моделирование. Markov Chain Monte-Carlo. Метод Метрополиса-Хастингса. \\
	Решение задачи 10}
\author{\emph{Яковлев Д.М.}\\\textsc{st095998@student.spbu.ru}}
\date{\today}


\begin{document}
	\maketitle
	\nocite{RomGu,ankorMLE}
	\section{Условия задачи}
	Исходная работа Хастингса в 1970 г. была посвящена моделированию стандартного нормально
	го распределения с использованием в качестве пробного распределения случайного блуждания
	с отдельными шагами, полученными из равномерного распределения на отрезке $(-\delta, \delta),~ \delta>0$.
	\begin{itemize}
		\item Исследуйте свойства алгоритма Метрополиса-Гастингса и получаемой цепи при разных
		значениях $\delta$ (например, $\delta = 0.1,1,10$ или другие). Какое значение будем оптимальным
		с точки зрения автокорреляции цепи? 
		\item Промоделируйте выборку из $N(0,1)$ с использованием других распределений для приращений: Коши и Лапласа (с параметром $\lambda = 1$). Сравните свойства получаемой марковской
		цепи. 
		\item Для трех выше обозначенных примеров случайного блуждания (с равномерными приращениями на интервале, из распределений Коши и Лапласа) исследуйте возможность получения вероятности перехода в следующее состояние цепи близкой к 0.25 за счет правильного
		подбора параметров. 
	\end{itemize}
	\bibliographystyle{ugost2008}
	\bibliography{MCMC}
\end{document}