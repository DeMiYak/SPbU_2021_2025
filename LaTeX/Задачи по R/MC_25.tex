\documentclass{article}

\usepackage[a4paper,
mag=1000, includefoot,
left=3cm, right=1.5cm, top=2cm, bottom=2cm, headsep=1cm, footskip=1cm]{geometry}
\usepackage[T2A]{fontenc}
\usepackage[utf8x]{inputenc}
\usepackage[english, main=russian]{babel}
\ifpdf\usepackage{epstopdf}\fi

\usepackage{amsmath,amssymb,amsthm,amscd,amsfonts}
\usepackage{mathdots}
\usepackage{graphicx}
\usepackage{algorithm}
\usepackage{algpseudocode}
\usepackage{algorithmicx}
\usepackage{bbm}

\newcommand\norm[1]{\left\|#1\right\|}
\DeclareMathOperator\R{\mathbb{R}}
\DeclareMathOperator\N{\mathbb{N}}
\DeclareMathOperator\rank{\textrm{rank}\,}


\title{Вычисление интегралов методом Монте-Карло\\
Решение задачи 25}
\author{\emph{Яковлев Д.М.}\\\textsc{st095998@student.spbu.ru}}
\date{\today}

\begin{document}
	\maketitle
	\section{Решение}

	Рассматривается многомерный интеграл из \cite{ankor} под номером 25
	\begin{equation}\label{eq: main}
		\dfrac{2}{\sqrt{\pi}}\iint\limits_{\substack{x>0\\0<t<cx}}x^{-p-1}e^{-t^2}~dx~dt.
	\end{equation}

	Последовательно будем решать несколько задач:
	\begin{enumerate}
		\item Убедиться, что интеграл сходится;
		\item Построить доверительный интервал по ЦПТ и для траекторий винеровского процесса;
	\end{enumerate}
	\subsection{}
	
		Обозначив интеграл \eqref{eq: main} как $J(c, p)$, найдём
		\begin{align*}
			J(c, p)=\dfrac{2}{\sqrt{\pi}}\left[~~\iint\limits_{\substack{x>0\\0<t<cx}}x^{-p-1}e^{-t^2}~dx~dt\overset{\text{т. Фубини}}{=}\int_{0}^\infty e^{-t^2}\left(\int_{t/c}^\infty x^{-p-1}~dx\right)~dt\overset{p > 0}{=}\int_0^\infty e^{-t^2}\dfrac{1}{p}\left(\dfrac{t}{c}\right)^{-p}~dt\right].
		\end{align*}
		Вынесем множитель $\dfrac{2c^p}{\sqrt{\pi}p}$ и рассмотрим сходимость интегрального выражения
		\begin{equation*}
			\int_0^\infty e^{-t^2}t^{-p}dt.
		\end{equation*}
		Заметим, что при $p\geqslant 1$ в правой окрестности $t=0$ (например, $[0,\varepsilon),~\varepsilon>0$):
		\begin{equation*}
			e^{-t^2}\rightarrow1,\int_0^\varepsilon e^{-t^2}t^{-p}~dt\rightarrow\int_0^\varepsilon t^{-p}dt = \infty,
		\end{equation*}
		Откуда получаем, что $0<p<1$. В таком случае
		\begin{align*}
			\int_0^\infty e^{-t^2}t^{-p}~dt=\left[t^2 = u,~t=u^{1/2},~dt=\dfrac{1}{2u^{1/2}}du\right]=\dfrac{1}{2}\int_0^\infty e^{-u}u^{\frac{-p-1}{2}}~du=\dfrac{1}{2}\Gamma\left(\dfrac{1-p}{2}\right).
		\end{align*}
		Тем самым, интегральное выражение
		\begin{equation*}
			J(c, p)=\dfrac{c^p}{\sqrt{\pi}p}\Gamma\left(\dfrac{1-p}{2}\right),~c>0,~0<p<1.
		\end{equation*}
	\subsection{}
	
%	Упростим $J(c, p)$:
%	\begin{equation*}
%		\dfrac{\sqrt{\pi}}{2}J(c, p)=\iint\limits_{\substack{x>0\\0<t<cx}}x^{-p-1}e^{-t^2}~dx~dt=\left[t = u/\sqrt{2},~dt=du/\sqrt{2}\right]=\dfrac{1}{\sqrt{2}}\iint\limits_{\substack{x>0\\0<u<cx\sqrt{2}}}x^{-p-1}e^{-u^2/2}~dx~du
%	\end{equation*}

	Подберём распределение $\mathcal{P}$. Для удобства вычислений, распределение $\mathcal{P}$ можно представить в виде двух независимых случайных величин. Пусть $\xi, \eta$ --- независимые случайные величины такие, что 
	\begin{align*}
		&\xi=|\varepsilon|,~\varepsilon~\sim N\left(0, \dfrac{1}{2}\right),~p_\xi(t)=\dfrac{2}{\sqrt{\pi}}e^{-t^2},~0<t,\\
		&\eta:~p_\eta(x)=C\begin{cases}
			0,&x\leqslant0\\
			e^{-x},&0<x\leqslant1\\
			x^{-p-1},&1<x
		\end{cases}.
	\end{align*}
	
	Определим нормирующий коэффициент $C$, вычислив интеграл от $p_\eta(x)$:
	\begin{align*}
		C\left[\int_{\R} d\mathcal{P}_\eta{dx}=\int_0^1e^{-x}~dx+\int_1^\infty x^{-p-1}~dx=(1-e^{-1}) + \dfrac{1}{p}\right]=1\Rightarrow C = \dfrac{1}{1-e^{-1}+\dfrac{1}{p}}.
	\end{align*}
	
	Само распределение $\mathcal{P}_\eta$ представляется как смесь распределений
	\begin{align*}
		&p_\eta(x)=q_1p_1(x)+q_2p_2(x),\\
		&q_1=C(1-e^{-1}),~p_1(x)= \mathbbm{1}_{(0,1]}(x)\dfrac{e^{-x}}{1-e^{-1}},~q_2=\dfrac{C}{p},~p_2(x)=\mathbbm{1}_{(1,+\infty)}(x)px^{-p-1},~q_1+q_2=1.
	\end{align*}
	
	Теперь определим обратные функции для полученных функций распределения
	\begin{align*}
		&p_1(x)\sim F_1(x)=\begin{cases}
			0,&x\leqslant0\\
			\dfrac{1-e^{-x}}{1-e^{-1}},&0<x\leqslant 1\\
			1,&1<x
		\end{cases}\Leftrightarrow x = -\log[1-(1-e^{-1})\alpha],~\alpha\in(0,1).\\
		&p_2(x)\sim F_2(x)=\begin{cases}
			0,&x\leqslant1\\
			1-x^{-p},&1<x<\infty
		\end{cases}\Leftrightarrow x = \sqrt[-p]{1-\alpha},~\alpha\in(0,1).
	\end{align*}
	 По теореме Радона-Никодима, существует
	\begin{equation*}
		m(\xi, \eta)=\dfrac{1}{C}\begin{cases}
			x^{-p-1}e^x,&0<x\leqslant1,~0<t<cx\\
			1,&1<x,~0<t<cx\\
			0,&\text{иначе}
		\end{cases}
	\end{equation*} 
	\bibliographystyle{ugost2008}
	\bibliography{MC}
\end{document}