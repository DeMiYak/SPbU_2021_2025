\documentclass[fleqn]{article}
\usepackage[a4paper,
mag=1000, includefoot,
left=2.5cm, right=1cm, top=2cm, bottom=2cm,
headsep=0.7cm, footskip=1cm]{geometry}
\usepackage{mathtext}
\usepackage{parskip}
\usepackage{setspace}
\linespread{1.2}

\usepackage[most]{tcolorbox}

\usepackage{enumitem}
\usepackage{amsmath,amssymb,amsthm,amscd,amsfonts}
\usepackage{cmap}
\usepackage[utf8]{inputenc}
\usepackage[T2A]{fontenc}
\usepackage[russian]{babel}
\usepackage{hyperref}
\hypersetup{
	colorlinks=true,
	linkcolor=blue,
	filecolor=magenta,      
	urlcolor=cyan,
	pdftitle={Overleaf Example},
	pdfpagemode=FullScreen,
}

\usepackage{euscript}
\usepackage{mathdots}
\usepackage{graphicx}
\usepackage[russian]{cleveref}
\usepackage{epstopdf}

\title{Программа минимум}
\date{}

\begin{document}
	\maketitle
	\begin{center}
		\textbf{Для получения оценки удовлетворительно необходимо ответить на один из случайных вопросов по каждой главе (всего 10 штук)}\hrule
	\end{center}
	\begin{enumerate}[start = 1, label =\textbf{Глава \arabic*.}]
		\item \textbf{Уравнения первого порядка, разрешенные относительно
			производной}
			\begin{enumerate}[label=1.\arabic*.]
				\item Определение решения дифференциального уравнения, область определения
				\item Понятие внутреннего (граничного, смешанного) решений; частное, особое и полного внутреннего (граничного, смешанного) решение. 
				\item Лемма о записи решения в интегральном виде (формулировка)
				\item Определение граничной и внутренней задачи Коши 
				\item Определение теоремы Пеано (внутреннее решение), построение треугольника Пеано
				\item Продолжимость решения в точку (на границу), за точку (за границу)
				\item Теорема о продолжимости решения на границу: формулировка и идея доказательства
				\item Лемма о продолжимости решения за границу отрезка
				\item Теорема о поведении интегральной кривой полного внутреннего решения
				\item Определение точки (не)единственности, область (не)единственности
				\item Теорема о локальной единственности решения внутренней задачи Коши
				\item Теорема о единственности в области; слабая
				\item Определение общего решения, теорема о существовании общего решения
				\item Определение поля направлений и изоклин
				\item Ломаные Эйлера: их построение (п. 2)
				\item Определение $\varepsilon$-решения, лемма о ломаных Эйлера в роли $\varepsilon$-решения (формулировка)
				\item Лемма Арцела-Асколи (формулировка) и сопутствующие ей определения (о функциональных последовательностях)
				\item Граничная задача Коши в $\mathbb{R}^2$. Упрощение через замену (п.3)
				\item Определение верхне/нижнеграничных функций
				\item Определение надграфиков, подграфиков и остальных случаев верхне/нижнеграничных функций
				\item Граничный треугольник и граничный отрезок Пеано: построение.
				\item Теорема о существовании решений граничной задачи Коши (формулировка)
				\item Теорема об отсутствии решений граничной задачи Коши (формулировка)
				\item Лемма о продолжимости решений на отрезок Пеано (формулировка и идея док-ва)
				\item Теорема о локальной единственности решения внутренней задачи Коши (формулировка)
				\item Лемма Гронуолла (формулировка), важное следствие из леммы
				\item Понятие условия Липшица
				\item Теорема о множестве единственности (п.3, пункт 4) (формулировка)
				\item Лемма о поведении решений на компакте $\bar{A}$ (формулировка)
				\item Теорема о существовании общего решения (формулировка и идея док-ва в три этапа)
				\item Теорема о дифференцируемости общего решения (формулировка)
			\end{enumerate}
		\item \textbf{Уравнения первого порядка в симметричной форме}
			\begin{enumerate}[label=2.\arabic*.]
				\item 
			\end{enumerate}
		\item \textbf{Нормальные системы ОДУ}
			\begin{enumerate}[label=3.\arabic*.]
				\item 
			\end{enumerate}
		\item \textbf{Линейные уравнения высокого порядка}
			\begin{enumerate}[label=4.\arabic*.]
				\item 
			\end{enumerate}
		\item \textbf{Линейные системы}
			\begin{enumerate}[label=5.\arabic*.]
				\item 
			\end{enumerate}
		\item \textbf{Автономные системы}
			\begin{enumerate}[label=6.\arabic*.]
				\item 
			\end{enumerate}
		\item \textbf{Теория устойчивости движения по Ляпунову}
			\begin{enumerate}[label=7.\arabic*.]
				\item 
			\end{enumerate}
		\item \textbf{Теория нормальных форм Пуанкаре}
			\begin{enumerate}[label=8.\arabic*.]
				\item 
			\end{enumerate}
		\item \textbf{Интегрирование основных типов уравнений первого порядка}
			\begin{enumerate}[label=9.\arabic*.]
				\item 
			\end{enumerate}
		\item \textbf{Интегрирование уравнений высокого порядка и систем}
			\begin{enumerate}[label=10.\arabic*.]
				\item 
			\end{enumerate}
	\end{enumerate}
\end{document}